% set up document type -- article
\documentclass[12pt]{article}

% import packages
\usepackage{hyperref}
\usepackage{listings}
\usepackage{amsfonts}

% set up the content page feature
\hypersetup{
    colorlinks=true,
    linkcolor=blue,
}

% set up the listings package feature
\lstset{
  frame=single,
  breaklines=true
}


\begin{document}

% create a title page
\title{CS2ME3 Assignment 1 report}
\author{Mingnan Su \\ macID: sum1}
\maketitle

% set up the hyperlink for {Return to content page}
\addtocontents{toc}{\protect\hypertarget{toc}{}}
% create a content page
\tableofcontents

\newpage
% 1 present my codes
\section{My Codes}

  % 1.1 pointADT.py
  \subsection{pointADT.py}
  \lstinputlisting{../src/pointADT.py}

  % 1.2 lineADT.py
  \subsection{lineADT.py}
  \lstinputlisting{../src/lineADT.py}

  % 1.3 circleADT.py
  \subsection{circleADT.py}
  \lstinputlisting{../src/circleADT.py}

  % 1.4 deque.py
  \subsection{deque.py}
  \lstinputlisting{../src/deque.py}

  % 1.5 testCircleDeque.py
  \subsection{testCircleDeque.py}
  \lstinputlisting{../src/testCircleDeque.py}

  % 1.6 Makefile
  \subsection{Makefile}
  \lstinputlisting{../Makefile}


% leave some white space
\vskip 3cm
% put the link to content page at the front
\hyperlink{toc}{Return to content page.}

% 2 present my partner's circleADT.py codes
\section{Partner Codes}

  % 2.1 CircleADT.py
  \subsection{CircleADT.py}
  \lstinputlisting{../srcPartner/circleADT.py}


% leave some white space
\vskip 5cm
% put the link to content page at the front
\hyperlink{toc}{Return to content page.}

% 3 the result of testing my files
\section{The result of testing my files}

  % 3.1 test result given
  \subsection {Compilation result}
  \begin{lstlisting}[frame=single]
  ....................................................
  ----------------------------------------------------
  Ran 52 tests in 0.002s

  OK
  \end{lstlisting}

  % 3.2 rational for test case selection
  \subsection {rational for test cases}
  For every method in an ADT, and for every access program
  in deque.py, I create a normal case for each procedure.

  
  Then, I check for exceptions for every procedure which have
  exceptions stated in the specification.


  Also, I check for the boundary cases.
  \begin{itemize}
    \item For pointADT, I check for the distance between two 
  overlapped points, and check for rotation of a point with 
  0, 360, negative degrees.
    \item For lineADT, I check for the length and the middle 
  point of the line with two overlapped points. I also check 
  for the rotation of the line with 0, 360, negative degrees.
    \item For circleADT, I check for the intersection between
  two nested circle and two boundary touched circle.
    \item For deque, I check if the size is zero after we 
  initialize a deque. And I check for the result of disjoint
  method with only one circlei in the deque.
  \end{itemize}


% leave some white space
\vskip 5cm
% put the link to content page at the front
\hyperlink{toc}{Return to content page.}

% 4 The results of testing my files along with my partner's
\section{The result of testing partner's files}

  % 4.1 test result given
  \subsection {Compilation result}
  \begin{lstlisting}[frame=single]
  ....................................................
  ----------------------------------------------------
  Ran 52 tests in 0.002s

  OK
  \end{lstlisting}

  % 4.2 the number of passed and failed test cases
  \subsection {the number of passed and failed test cases}
  There is 52 test cases in total, all passed.

  % 4.3 failed test cases
  \subsection {details of failed test cases}
  All the test cases are passed, including the boundary cases
  I stated above. 

  % 4.4 summary of overal results
  \subsection {Summary}
  All test passed. As shown in my partner's code conmment, 
  he/she already considered the boundary cases I stated in
  the test file. The implementation of the programs also 
  does not make any big differences between us. Hence all
  the test cases are passed.


% leave some white space
\vskip 5cm
% put the link to content page at the front
\hyperlink{toc}{Return to content page.}

% 5 Discussion of test results and what I learned
\section{Discussion of test results}

  % 5.1 What I leanred by doing this assignment
  \subsection {What I learned}
  This assignment gives us an opportunity to experience 
  the real world programming implementation. According
  to the mathematical specification, we can learn how to
  translate the math expression into programming language.
  By going through the process of sepecification
  implementation, we get a better understanding of the
  advantages of module interface specification.

  % 5.2 List any problem I found
  \subsection {Any problem I found with: }
  \begin{enumerate}
    % my program
    \item My program
    \newline In my program, I found a problem that my
    implementation of the specification is not as clear
    as possible. Some part of my code may be confusing
    to the reader who reads my code. Also, when I was
    planning and writing my code, I didn't take the 
    performance into my consideration. Thus, the true
    implementation may result in bad performance.

    % partner's module
    \item Partner's module
    \newline In my partner's module, all the test cases
    pass through my test file. The true implementation
    do not have much difference between he/her and me.
    So the problem I found with my partner's module 
    might be similar to me (as stated above). The only
    problem that he/her have but I do not have is about
    state varibles. I make the state variable in my ADT
    private to prevent unexpected outernal changes in 
    ADT objects state variable.

    % the specification of the modules
    \item the specification of the modules
    \newline The specification of the modules might can 
    be improved by omitting unnecessary methods. Also, 
    every specification of the module in this assignment 
    depends on one or more other modules, which means as 
    long as one module is detected with error, abstracted 
    object deque will crash.
  \end{enumerate}

  % 5.3 formal MIS in a2 vs. informal specifiation in a1
  \subsection {formal MIS compare to informal sepecification}
  Formal module interface specification have better format
  so that the reader could easily understand what the program
  is going to achieve compared to informal specification. In
  assignment 1, we are given informal specification which
  present the requirement of the module in plain words. Without
  formal mathematical expression, some idea may end up being
  ambiguous or igored. On the other hand, the formal module
  interface specification in assignment 2 helps avoiding 
  ambiguities and have a clear table to inform what programmer
  should implement.

  % 5.4 advantages using a testing framework
  \subsection {testing  framework (pyUnit) advantages}
  Testing framewrok, in this example pyUnit, helps programmer
  write a better test program. Some feature, such as setUp
  method, avoid code duplicate in a test program and speed 
  up the programmer to write code. 


% leave some white space
\vskip 5cm
% put the link to content page at the front
\hyperlink{toc}{Return to content page.}

% 6 Specification for last two access programs as LaTeX equations
\section{Specification for totalArea(), averageRadius()}

  % 6.1 totalArea() specification
  \subsection{totalArea()}
  \begin{displaymath}
  out:= +(i:\mathbb{N}|i\in[0..|s|-1]:s[i].\textrm{area}())
  \end{displaymath}

  % 6.2 averageRadius() specification
  \subsection{averageRadius()}
  \begin{displaymath}
  out:= \frac{+(i:\mathbb{N}|i\in[0..|s|-1]:s[i].\textrm{rad}())}
             {|s|}
  \end{displaymath}


% leave some white space
\vskip 5cm
% put the link to content page at the front
\hyperlink{toc}{Return to content page.}

% 7 Provide a critique of the Circle Module's interface
\section{Critique of Circle Module's interface}

  % comment specificaly
  Determine if the exported access program provide is:
  \begin{itemize}
    \item consistent
    \newline This module is consistent. The naming convetion
    (circle ADT represents a circle) matches common people's
    perspectives. A circle is located by a point and radius,
    which is also easy to understand.
    \item essential
    \newline A method called intersect in circleADT, could be
    solved by using another method connection. The programmer
    can evaluate the length of the connection, and then compare
    to the sum of two circle's radius, the comparison result
    could indicates if these two circles intersect. Hence, 
    circleADT is not essential in some senses. However, intersect
    method do a lot of great job by simplifying the code, since
    we are going to implement deque.disjoint() after. 
    \item general
    \newline Circle module's interface could be considered as
    general since it is very easy to understand and mathces
    common senses. But intersect method ignore the boundary cases
    (when a small circle is inside the big circle, this module
    considered these two circle intersect), which might be hard
    for people to understand.
    \item minimal
    \newline Circle module have two methods (rad and area) which
    both do not take any input and output a real number. This 
    situation might opposes the idea of minimal. But other than
    that, this module do a great job on minimize the program.
    \item opaque
    \newline In circle module's interface, it uses pointADT and
    lineADT. This specification hides the details of these two
    modules and implement them directly. For example, when we 
    want to create a new circle, we only need to give a real
    number (radius) and a PointT (a point). What the PointT should
    look like is not stated in the module. Hence, circle module 
    could be considered as opaque since it hides the information
    of the module it used.
  \end{itemize}


% leave some white space
\vskip 10cm
% put the link to content page at the front
\hyperlink{toc}{Return to content page.}

% 8 Discussion of output of Deq_disjoint() when only one circle
\section{Output of Deq.disjoint() discussion}

  % 8.1 What is the output of specification?
  \subsection{Output generated by specification (with explanation)}
  According to specification, the output should be false, cause we
  only have one circle in the deque, which means the condition will
  never meet.

  % 8.2 Does this answer make sense? 
  \subsection{Is output above make sense?}
  This answer does not make sense, since if there is only one circle
  in the deque, the only circle does not have any chance intersect
  with another circle. The output should be true.

  % 8.3 Is it the same result of my code?
  \subsection{Is output above the same as the result calculated by my code?}
  No. The result generated by the program is true.

\end{document}
\grid
